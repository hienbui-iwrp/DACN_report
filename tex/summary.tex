% !TeX root = ..\main.tex
%%%%%%%%%%%%%%%%%%%%%%%%%  
\section{Hạn chế của đề tài}
\hspace{0.5cm} Khi thực hiện chuyển đổi BPMN thành BPEL, các User Task trong BPMN sẽ được chuyển thành Human Task trong BPEL, tuy nhiên việc thực hiện các Human Task này yêu cầu phải thông qua ứng dụng BPM Worklist và được thực hiện bởi tài khoản được cung cấp từ Weblogic Server của Oracle và không thể được thực hiện từ xa (thông qua Http API), do đó việc thực hiện các Human Task chỉ phù hợp với các hệ thống quản lý nội bộ và làm việc trực tiếp với ứng dụng từ Oracle. \\

Vì vậy, trong đề tài này, nhóm chỉ thực hiện xây dừng các quy trình BPEL mà trong đó hệ thống thực hiện tự động hoàn toàn mà không có các hành động xen giữa quy trình của con người để phù hợp với hệ thống bán hàng.


%%%%%%%%%%%%%%%%%%%%%%%%%
\section{Những nhiệm vụ đã hoàn thành cho đề tài}
Trong giai đoạn đồ án chuyên ngành, nhóm đã hiện thực được những công việc như sau:
\begin{itemize}
    \item Tìm hiểu về quy trình nghiệp vụ và các công cụ sử dụng để mô hình hóa quy trình nghiệp vụ.
    \item Tìm hiểu về các chuyển đổi từ lược đồ BPMN sang BPEL để áp dụng cho tự động hóa quy trình nghiệp vụ.
    \item Phân tích yêu cầu chức năng và phi chức năng của hệ thống.
    \item Xây dựng và mô hình hóa được quy trình nghiệp vụ cho hệ thống.
    \item Hoàn thành thiết kế kiến trúc cho hệ thống: lược đồ use-case, cơ sở dữ liệu, lược đồ class, giao diện người dùng.
    \item  Lựa chọn công nghệ để hiện thực front-end và back-end cho hệ thống.
\end{itemize}




%%%%%%%%%%%%%%%%%%%%%%%%%
% \subsection{Phân chia công việc}
