% !TeX root = ..\main.tex
\section{Lựa chọn công nghệ}

    \subsection{Front-end}
    
    \subsection{Back-end}
\par Ở phần back-end, nhóm sử dụng ngôn ngữ Golang để hiện thực. Đây là một ngôn ngữ mới và hiệu quả cho việc xây dựng hệ thống nói chung, và xây dựng hệ thống microservice nói riêng.
Những lợi ích mà Golang có bao gồm:
\begin{itemize}
    \item Golang là ngôn ngữ static-typed, kiểu dữ liệu của biến là cố định và được khai báo từ đầu. Do đó các lỗi xảy ra liên quan đến kiểu dữ liệu đều được phát hiện và ngăn chặn ở bước compile. Ngoài ra tính năng thu gom rác (Garbage collection) giúp Golang quản lý bộ nhớ an toàn và hạn chế lỗi xuất hiện.
    \item Golang là ngôn ngữ đơn giản và dễ học. Cú pháp đơn giản và trực quan của Golang giúp cho ngôn ngữ này được ưa thích và người dùng có thể học dễ dàng.
    \item Golang là ngôn ngữ phổ biến, được sử dụng rộng rãi. Ngôn ngữ này được tạo nên và phát triển bởi Google, cùng cộng đồng lớn, do đó việc học và tìm sự trợ giúp khi sử dụng Golang trở nên dễ dàng hơn.
    \item Thời gian build và thực thi thấp hơn. Golang được tối ưu để thời gian build và chạy ít hơn.
    \item Golang hỗ trợ mạnh việc lập trình concurrency. Điều này giúp các ứng dụng Golang có khả năng mở rộng lớn.
\end{itemize}
\par Qua đó, có thể thấy Golang có nhiều ưu điểm khi sử dụng. Với lợi thế về hiệu năng, khả năng mở rộng và độ tin cậy, Golang là lựa chọn thích hợp cho nhóm trong dự án này.