% !TeX root = ..\main.tex

\section{Front-end}
Để lựa chọn công nghệ phù hợp để hiện thực, nhóm sẽ thực hiên xem xét một số công nghệ phổ biến hiện nay

\subsection{React}
\hspace*{0.5cm} React là một thư viện Javascript dùng để xây dựng giao diện người dùng được phát triển bởi facebook\\

\textbf{Ưu điểm \cite{technologyFE}:}
\begin{itemize}
    \item Các component có thể tái sử dụng của React đảm bảo rằng các lập trình viên không phải viết đi viết lại cùng một đoạn codes
    \item Do tính phổ biến của react, việc tìm kiếm sự giúp đỡ, tài nguyên dành cho React là rất dễ dàng
\end{itemize}

\textbf{Nhược điểm \cite{technologyFE}:}
\begin{itemize}
    \item Các tài liệu không nhất quán.
    \item Không hỗ trợ SEO
\end{itemize}

\subsection{Vue.js}

\hspace*{0.5cm} Là framework của Javascript được phát triển bởi Google, là phiên bản tối giản của Angular.    \\


\textbf{Ưu điểm \cite{technologyFE}:}
\begin{itemize}
    \item Các thành phần có thể được tạo ra từ HTML, CSS và JS truyền thống. Chính vì thế nó giúp các lập trình viên dễ dàng tìm hiểu. Dễ học hơn so với React.
    \item Nhiều thư viện và công cụ hỗ trợ.
    \item Đơn giản, dễ học
\end{itemize}

\textbf{Nhược điểm \cite{technologyFE}:}
\begin{itemize}
    \item Tài liệu còn ít, do ra đời muộn
    \item Cộng động sử dụng ít hơn so với React
\end{itemize}

\subsection{Next.js}

\hspace*{0.5cm} Next.js là một framework được phát triển bổ sung các khả năng render phía máy chủ (SSR). Next.js xây dựng dựa trên nền tảng là thư viện React, do đó các ứng dụng Next.js tương tự như các ứng dụng từ React nhưng được thêm các tính năng bổ sung.\\


\textbf{Ưu điểm \cite{technologyNextAdvance}:}
\begin{itemize}
    \item Dễ hiểu và dễ sử dụng: Có nền tảng từ ReactJs nên không khó để học và sử dụng với cộng đồng sử dụng lớn
    \item Hỗ trợ SEO: Next.js được bổ sung khả năng SSR, giúp hỗ trợ cho SEO tốt hơn nhiều so với React
\end{itemize}

\textbf{Nhược điểm \cite{technologyNextAdvance}:}
\begin{itemize}
    \item Quá phức tạp và không cần thiết đối với các ứng dụng đơn giản
    \item Thời gian build hệ thống lớn hơn so với React
\end{itemize}

Vì nhóm xây dựng một hệ thống cửa hàng thời trang, do đó việc hỗ trợ SEO là một yếu tố không thể bỏ qua nên Next.js là công nghệ nhóm chọn để sử dụng. \\

Bên cạnh đó nhóm còn sử dụng thư viện components để hỗ trợ việc hiện thực hệ thống:

\subsection*{Ant Design}
\hspace*{0.5cm} Ant Design là một thư viện của React cung cấp rất nhiều component có tính thông dụng cao, đa dạng và đầy đủ để đáp ứng các nhu cầu để không cần phải định nghĩa thêm các base component nữa. Ant Design có một cộng đồng sử dụng rất lớn, hiện tại số lượt started ở trên github của Ant Design đã đạt hơn 83 nghìn lượt \cite{technologyAntdStar}.

\begin{figure}[!htp]
    \begin{center}
        \includegraphics[width=5cm]{img/Technology/antd.jpg}
    \end{center}
    \caption{Logo Ant Design \cite{technologyAntd}}
\end{figure}

\hspace*{0.5cm} Ưu điểm của Ant Design bao gồm:
\begin{itemize}
    \item Số lượng base component: Như đã giới thiệu, Ant Design cung cấp rất nhiều base component đa dạng và đầy đủ giúp ta không cần xài thêm thư viện giao diện khác nữa.
    \item Khả năng customize base component: Ant Design cung cấp rất nhiều props để lập trình viên có thể tùy biến linh hoạt theo nhu cầu.
    \item Dễ học, dễ hiểu: Ant Design cung cấp bộ tài liệu hướng dẫn rất chi tiết cùng với demo tương ứng giúp người dùng có thể dễ dàng hiểu rõ và sử dụng.
    \item Responsive: Ant Design hỗ trợ dàn layout tương thích trên nhiều thiết bị và loại màn hình khác nhau, vì vậy không cần phải xài thêm các thư viện khác để responsive.
    \item Cộng đồng sử dụng: Ant Design có cộng đồng sử dụng lớn giúp cho việc tìm kiếm giải đáp thắc mắc rất dễ dàng.
\end{itemize}


\section{Back-end}
\hspace{0.5cm}Ở phần back-end, nhóm sử dụng ngôn ngữ Golang để hiện thực. Golang là một ngôn ngữ mới, được phát triển bởi Google nhằm sử dụng cho các hệ thống scale lớn. Đây là ngôn ngữ hiệu quả cho việc xây dựng hệ thống nói chung, và xây dựng hệ thống microservice nói riêng.

\begin{figure}[!htp]
    \begin{center}
        \includegraphics[width=5cm]{img/Technology/Golang.png}
    \end{center}
    \caption{Logo Golang \cite{technologyGolang}}
\end{figure}

Những lợi ích mà Golang có bao gồm\cite{technologyGolangReasons}:
\begin{itemize}
    \item Golang là ngôn ngữ static-typed, kiểu dữ liệu của biến là cố định và được khai báo từ đầu. Do đó các lỗi xảy ra liên quan đến kiểu dữ liệu đều được phát hiện và ngăn chặn ở bước compile. Ngoài ra tính năng thu gom rác (Garbage collection) giúp Golang quản lý bộ nhớ an toàn và hạn chế lỗi xuất hiện.
    \item Golang là ngôn ngữ đơn giản và dễ học. Cú pháp đơn giản và trực quan của Golang giúp cho ngôn ngữ này được ưa thích và người dùng có thể học dễ dàng.
    \item Golang là ngôn ngữ phổ biến, được sử dụng rộng rãi. Ngôn ngữ này được tạo nên và phát triển bởi Google, cùng cộng đồng lớn, do đó việc học và tìm sự trợ giúp khi sử dụng Golang trở nên dễ dàng hơn.
    \item Thời gian build và thực thi thấp hơn. Golang được tối ưu để thời gian build và chạy ít hơn.
    \item Golang hỗ trợ mạnh việc lập trình concurrency. Điều này giúp các ứng dụng Golang có khả năng mở rộng lớn.
\end{itemize}
\hspace{0.5cm}Qua đó, có thể thấy Golang có nhiều ưu điểm khi sử dụng. Với lợi thế về hiệu năng, khả năng mở rộng và độ tin cậy, Golang là lựa chọn thích hợp cho nhóm trong dự án này.