% !TeX root = ..\main.tex
\section{Tổng quan}
%%%%%%%%%%%%%%%%%%%%%%%%%
\subsection{Giới thiệu đề tài}

\hspace{0.5cm}Trong thời đại công nghiệp 4.0 hiện nay, khi truy cập dữ liệu thời gian thực được cho phép và sự bùng nổ của không gian mạng, quá trình sản xuất được tiếp cận và cải thiện tốt hơn bởi việc sử dụng công nghệ. Việc kết nối vật lý với kỹ thuật số cho phép bộ phận, đối tác, khách hàng, nhà cung cấp, sản phẩm và con người có thể kết nối tốt hơn. Công nghiệp 4.0 đang giúp các công ty, doanh nghiệp có thể dễ dàng hợp tác và chia sẻ dữ liệu giữa các khách hàng, nhà sản xuất, nhà cung cấp và các bên khác trong chuỗi cung ứng. Nó cải thiện năng suất và khả năng cạnh tranh, cho phép chuyển đổi sang nền kinh tế kỹ thuật số và cung cấp cơ hội để đạt được tăng trưởng kinh tế và bền vững. Với những lợi ích trong thời đại công nghiệp 4.0 mang đến như vậy, việc chuyển đổi số của các doanh nghiệp là không thể tránh khỏi. Với khả năng cập nhật dữ liệu nhanh chóng, các nhà quản lý hay chủ doanh nghiệp có thể nắm bắt khía cạnh khác nhau của doanh nghiệp 1 cách nhanh chóng và có thể tìm cách để cải thiện. Mà trong 1 doanh nghiệp khi hoạt động, để đạt được những mục tiêu kinh doanh thì phải trải qua chuỗi các hoạt động khác nhau và đó chính là quy trình nghiệp vụ.\\

Quy trình nghiệp vụ là một thành phần quan trọng trong quá trình hoạt động của doanh nghiệp, nó đảm bảo các hoạt động được thực hiện có trình tự và sẽ đạt được kết quả đúng với mục tiêu. Các quy trình nghiệp vụ đại diện cho một trong những tài sản cốt lõi của các tổ chức vì nhiều lý do khác nhau. Chúng có tác động trực tiếp đến sức hấp dẫn của sản phẩm và dịch vụ, ảnh hưởng đến trải nghiệm của khách hàng và cuối cùng là ảnh hưởng đến doanh thu của doanh nghiệp. Các quy trình phối hợp các nguồn lực của công ty để đáp ứng các nhu cầu bên ngoài này, do đó các quy trình này là yếu tố chính xác định chi phí để phục vụ và hoạt động hiệu quả. Cụ thể, chúng xác định các nhiệm vụ, công việc và trách nhiệm, bằng cách này, ta có thể định hình công việc trong tương lai của mọi nhân viên và máy móc trong quy trình kinh nghiệp. Các quy trình là hệ thống quan trọng trong tổ chức và trong mạng lưới cung ứng liên tổ chức. Do đó, bất kỳ lỗi quy trình nào cũng có thể khiến hoạt động của công ty và toàn bộ hệ sinh thái bị đình trệ. Các quy trình xác định tiềm năng và tốc độ của một tổ chức để thích ứng với hoàn cảnh mới và tuân thủ số lượng yêu cầu lập pháp đang tăng lên nhanh chóng.\\

Nhu cầu ngày càng tăng về toàn cầu hóa, tích hợp, tiêu chuẩn hóa, đổi mới, nhanh nhẹn và hoạt động hiệu quả, cùng với các cơ hội do công nghệ kỹ thuật số mang lại, cuối cùng đã làm tăng nhu cầu cải thiện  cũng như thiết kế toàn bộ quy trình nghiệp vụ mới. Một bộ công cụ, kỹ thuật và phương pháp được sinh ra để hỗ trợ tất cả các giai đoạn của vòng đời quy trình nghiệp vụ đã xuất hiện trong hai thập kỷ qua. Nó được gọi là "Quy trình quản lý nghiệp vụ (BPM)". BPM hợp nhất rất nhiều công cụ và cách tiếp cận đến từ nhiều lĩnh vực khác nhau, bao gồm kỹ thuật công nghiệp, quản lý hoạt động, quản lý chất lượng, quản lý nguồn nhân lực, quản trị doanh nghiệp, khoa học máy tính và kỹ thuật, hệ thống thông tin.\\

Khi các quy trình nghiệp vụ đã được xây dựng và được thực hiện, thì trong quá hình hiện thực có thể xảy ra sai sót bởi con người hay những quy trình có thể cần được cải thiện và khi sửa đổi quy trình, doanh nghiệp cần tiêu tốn rất nhiêu thời gian cũng như nguồn lực để xây dựng lại ứng dụng điều phối quy trình. Từ đây, nhu cầu áp dụng công nghệ vào việc tự động hóa các quy trình được sinh ra. \textbf{Tự động hóa quy trình nghiệp vụ (BPA)} là việc sử dụng phần mềm để tự động hóa các giao dịch kinh doanh nhiều bước, có thể lặp lại. Trái ngược với các loại tự động hóa khác, các giải pháp BPA có xu hướng phức tạp, được kết nối với nhiều hệ thống công nghệ thông tin (CNTT) và được điều chỉnh cụ thể theo nhu cầu của một tổ chức. Các tổ chức thường áp dụng BPA như một phần của chiến lược chuyển đổi kỹ thuật số, nhằm hợp lý hóa quy trình làm việc của họ và hoạt động hiệu quả hơn. Tự động hóa quy trình nghiệp vụ có thể giải phóng thời gian và nguồn lực cho doanh nghiệp rất nhiều.\\

Với các yêu cầu về thiết kế và mô hình hóa quy trình kinh doanh như vậy, mô hình và ký hiệu mô hình hóa quy trình nghiệp vụ (BPMN) là ngôn ngữ hướng đồ thị được sinh ra và cung cấp cho các doanh nghiệp khả năng hiểu các quy trình kinh doanh nội bộ của họ dưới dạng ký hiệu đồ họa và sẽ cung cấp cho các tổ chức khả năng truyền đạt các quy trình này theo cách tiêu chuẩn. Hơn nữa, ký hiệu đồ họa sẽ tạo điều kiện cho sự hiểu biết về hiệu suất hợp tác và giao dịch kinh doanh giữa các tổ chức. Điều này sẽ đảm bảo rằng các doanh nghiệp sẽ hiểu chính họ và những người tham gia vào hoạt động kinh doanh của họ, đồng thời sẽ cho phép các tổ chức điều chỉnh nhanh chóng các tình huống kinh doanh nội bộ và B2B mới. Khi thiết kế và cải thiện quy trình kinh doanh sử dụng BPMN thì đến khi thực hiện, chúng ta cần sử dụng đến BPEL(ngôn ngữ thực thi quy trình doanh nghiệp) thể có thể hiện thực nó, được dùng bởi các lập trình viên và nhà phân tích kỹ thuật. Trong khi BPMN thể hiện các ký hiệu để cho các phía về mặt business đều hiểu thì BPEL lại là ngôn ngữ thực thi để lập trình viên có thể hiện thực những mô tả từ các ký hiệu trong BPMN. Hiện nay có 1 số công cụ hỗ trợ chuyển đổi từ BPMN sang BPEL như Oracle Business Process Analysis Suite. Với các công nghệ và kỹ thuật ở trên thì ta có thể kết hợp lại để thực hiện BPA, khi doanh nghiệp muốn cập nhật hay cải thiện về quy trình kinh doanh thì có thể cập nhật trên các lược đồ BPMN, từ đầu chúng ta có thê sử dụng các công cụ để chuyển đổi thành lượt đồ BPEL, sau khi được chuyển đổi thì có thể cập nhật vào hệ thống để hệ thống cập nhật quy trình mới, từ đó tiết kiệm được lượng lớn chi phí khi phải thuê một đội ngũ công nghệ thông tin để cập nhật lại hệ thống khi có thay đổi. Qua các nội dung đã được trình bày ở trên, nhóm sẽ thực hiện áp dụng tự động hóa quy trình nghiệp vụ cho một hệ thống cửa hàng thời trang trong đề tài: "Xây dựng ứng dụng quản lý hệ thống cửa hàng thời trang dựa trên việc tự động hóa quy trình nghiệp vụ".\\
%%%%%%%%%%%%%%%%%%%%%%%%%
\subsection{Mục tiêu đề tài}
\par Mục tiêu của đề tài là áp dụng tự động hóa quy trình nghiệp vụ vào trong một hệ thống thực tế. Việc vận hành hệ thống cần đảm bảo các yêu cầu đặt ra ban đầu, bao gồm các tính năng đề ra. Về phần quy trình nghiệp vụ, nghiệp vụ của hệ thống được quản lý và tự động hóa bằng phương pháp tự động hóa quy trình nghiệp vụ. Việc thiết kế và cập nhật nghiệp vụ sẽ được thuận tiện và thân thiện với cả người dùng nghiệp vụ lẫn lập trình viên với BPMN và việc cập nhật nghiệp vụ vào hệ thống sẽ được tự động với sự trợ giúp của BPEL.

%%%%%%%%%%%%%%%%%%%%%%%%%
\subsection{Phạm vi đề tài}
\par Phạm vi của đề tài này bao gồm:
\begin{itemize}
	\item Thiết kế và hiện thực một hệ thống bán hàng thời trang, bao gồm các tính năng mua / bán hàng trực tuyến và trực tiếp, và các tính năng quản lý hoạt động của hệ thống bán hàng.
	\item Thiết kế quy trình nghiệp vụ của doanh nghiệp bằng phương pháp tự động hóa quy trình nghiệp vụ, biểu diễn quy trình nghiệp vụ bằng BPMN và thực thi nghiệp vụ bằng BPEL.
	\item Áp dụng BPEL vào việc tự động hóa quy trình nghiệp vụ của hệ thống.
\end{itemize}

%%%%%%%%%%%%%%%%%%%%%%%%%
\subsection{Ý nghĩa đề tài}
\par Trong thực tế, quy trình nghiệp vụ của doanh nghiệp thường lớn và phức tạp, đòi hỏi sự tham gia của nhiều thành phần có các mối liên hệ phức tạp với nhau. Một sự thay đổi nghiệp vụ đòi hỏi nhiều thời gian và công sức để áp dụng vào hệ thống thực tế. Đối với phần lập trình và phát triển hệ thống, lập trình viên cần thay đổi mã nguồn ở nhiều nơi để đáp ứng đúng với thay đổi của nghiệp vụ, và cần đảm bảo sau khi thay đổi hệ thống vừa chạy đúng với nghiệp vụ mới, vừa đáp ứng các yêu cầu chức năng và phi chức năng đã đặt ra từ đầu. Với cách tiếp cận này, việc thay đổi và cập nhật quy trình nghiệp vụ sẽ tốn nhiều thời gian khi phải thay đổi mã nguồn ở nhiều phần trong hệ thống, sẽ tốn công sức để đảm bảo hệ thống đáp ứng đúng như yêu cầu sau cập nhật, và dễ có rủi ro nếu việc tìm và thay đổi mã nguồn vẫn chưa triệt để.

\par Với sự giúp đỡ của việc \textbf{Tự động hóa quy trình nghiệp vụ}, việc quản lý quy trình nghiệp vụ sẽ trở nên dễ dàng hơn rất nhiều. Quy trình nghiệp vụ có thể được biểu diễn dưới dạng mô hình BPMN - một mô hình có thể hiểu được bởi người dùng nghiệp vụ và lập trình viên. Việc thay đổi và cập nhật quy trình nghiệp vụ sẽ được tối ưu và tự động với \textbf{Ngôn ngữ thực thi quy trình doanh nghiệp - BPEL}. Các thành phần của hệ thống tương ứng với các bước trong quy trình nghiệp vụ sẽ được kết nối đến các thành phần của BPEL, do đó khi có sự thay đổi của quy trình nghiệp vụ, chỉ cần cập nhật lại BPEL và BPEL sẽ gọi thứ tự thực thi nghiệp vụ mới theo đúng như nó được cập nhật. Do đó, lập trình viên không cần tìm và sửa mã ở khắp nơi trong mã nguồn, mà chỉ cần cập nhật BPEL và đảm bảo kết nối đúng các thành phần BPEL đến các thành phần trong hệ thống. Do BPEL là một ngôn ngữ thực thi, nó có thể được deploy và chạy, nên khi có sự thay đổi nghiệp vụ ở BPMN và BPEL, việc cập nhật nghiệp vụ sẽ nhanh và tự động mà không yêu cầu nhiều công sức của con người.

%%%%%%%%%%%%%%%%%%%%%%%%%
\subsection{Đóng góp của đề tài}
\par Đề tài mong muốn mang đến một cách tiếp cận hiệu quả cho việc thiết kế và quản lý quy trình trong mỗi doanh nghiệp. Sau khi hoàn thành, đề tài muốn cho thấy cách tiếp cận bằng Tự động hóa quy trình nghiệp vụ sẽ giảm đáng kể thời gian và nguồn lực trong việc quản lý quy trình nghiệp vụ, và sẽ mang lại hiệu quả lớn cho doanh nghiệp trong việc vận hành hệ thống của mình.

%%%%%%%%%%%%%%%%%%%%%%%%%   
\subsection{Phân chia công việc}
